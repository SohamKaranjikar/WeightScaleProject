%%%%%%%%%%%%
%PAGE SETUP%
%%%%%%%%%%%%
\documentclass[letterpaper,11pt]{article}
\usepackage[margin=1.0in]{geometry}


%%%%%%%%%
%IMPORTS%
%%%%%%%%%

\usepackage{titlesec}
\usepackage[usenames,dvipsnames]{color}
\usepackage[pdftex]{hyperref}
\usepackage{enumitem}
\usepackage{listings}

\usepackage{graphicx}
\graphicspath{{./Images/}}

\usepackage[T1]{fontenc}

%%%%%%
%MISC%
%%%%%%
\newcommand{\myLink}[2]{\href{#1}{\color{blue}\underline{\smash{\texttt{#2}}}}}
\newcommand{\myURL}[1]{\myLink{https://#1}{#1}}

\title{CS 431: Weight Scale Project}
\author{Ivan Johnson \and Soham Karanjikar}
\date{2020-05-01}


%%%%%%%
%BEGIN%
%%%%%%%
\begin{document}

\maketitle

\newpage
\section{Introduction}
The main motivating factor behind this project was to make a scale that could
take the weight readings and upload them the data to our
\myLink{https://github.com/Ivan-Johnson/LifeLogServer}{private servers}. As much
of this is beyond the scope of what was taught in class, however, for the
purposes of this class the project is limited to the much simpler goal of simply
measuring and displaying the force that is applied to a sensor.

Besides the learning experience of building the system ourselves, the main
benefit of this approach over buying a standard smart scale is that it gives us
complete control of the system. This alleviates many privacy concerns and allows
for the possibility of integrating the scale with arbitrary
\myLink{https://github.com/Ivan-Johnson/LifeLogServer}{health tracking systems}.

This was a non-trivial task requiring that we research the construction of
commercial bathroom scales, choose the most practical parts for our own design,
build then test the design, make the software to run the system, and finally
optimize the software to minimize power usage. This process enhanced our
learning experience in the course by giving us a more hands-on experience with
how embedded systems are designed and used.

The initial plan was to use an existing scale and try to take readings through
communicating with the micro controller on the scale, however, this proved to be
more work than creating a new scale. The components, procedure, results,
limitations, and potential improvements are explained in this paper. The
appendices contain our original proposal, our code, and a list of what we used
for this project.

\section{Components}
\subsection{Microcontroller}
The board used was an Arduino Uno, a hobbyist development board used in many
projects at their preliminary stage, with the MCU being an ATMEGA4809. The
reason for choosing this board was because of the ease of using it and the
libraries already available for communicating with various chips. Furthermore,
the Arduino IDE is also very user friendly and allowed us to quickly test
code. We could have potentially used a cheaper and smaller option such as an
ESP32, but there were Arduinos already present in the lab which allowed us to
start immediately.

\subsection{Load Cells and Amplifier Chip}
While researching methods to communicate with a controller on an existing scale
we found that load cells and their controller chip are very cheap and easy to
use. This is the reason we decided to build our own scale rather than tapping
into a pre-made scale.

The chip used for reading values from load cells is the HX711, an amplifier chip
specifically designed for load cells.

The bundle purchased came with four load cells included in it. These load cells
had an individual rating of 50Kg each allowing us to make a scale that could
provide accurate readings of masses up to 200Kg. The load cells were generic and
came with no model/serial number or brand.

\section{Procedure}
\subsection{Wiring}
Load cells use strain gauges to measure weight. A strain gauge contains a wire
carefully shaped such that its resistance varies when the strain gauge undergoes
elastic deformation. To form the load cell, the strain gauge is attached to a
piece of metal in such a way that the strain gauge can be used to measure the
deformation of the metal and infer the force that is being applied to the load
cell.

The load cells are wired according to the specification given to use in a
Wheatstone Bridge configuration, as shown in figure \ref{img:banggood}.

\begin{figure}[h]
  \centering
  \includegraphics[scale=.5]{loadcellwiring}
  \caption{Image obtained from
    \myLink{https://www.banggood.com/4pcs-DIY-50KG-Body-Load-Cell-Weight-Strain-Sensor-Resistance-With-HX711-AD-Module-p-1326815.html}{Bang
      Good}}
  \label{img:banggood}
\end{figure}

Since the change in resistance is quite small, we a HX711 chip to amplify the
change. This chip also has ADC included in it which allows us to directly
communicate with it using the serial protocol given in its documentation. The
whole system's wiring diagram is provided in figure \ref{img:wiring}.

\begin{figure}[h]
  \centering
  \includegraphics[scale=.3]{sytemwiring}
  \caption{Image obtained from
    \myLink{https://circuitjournal.com/50kg-load-cells-with-HX711}{Circuit Journal}}
  \label{img:wiring}
\end{figure}

The Arduino uses its 5V out to power the HX711 chip and two GPIOs to handle the
communication, one for clock and one for data.

\subsection{Software}
All the code was written in the Arduino IDE as a sketch; the code is available
in section \ref{apx:code} of the appendix. We only had one external dependency,
a \myLink{https://github.com/bogde/HX711}{HX711 library}. This library allowed
us to communicate with the amplifier chip quite easily as the low level
communication was already written for us. Figure \ref{img:stateflow} shows a
state diagram of our software.

\begin{figure}[h]
  \centering
  \includegraphics[scale=.7]{Flow}
  \caption{The state diagram of our code}
  \label{img:stateflow}
\end{figure}

The detection of HX711 is done by using the "is\_ready()" function from the
library which just checks if there is valid communication. Then the system
starts polling to read values from the chip, if they are above a certain
threshold (2Kg in our case) then the value is displayed on the console. If the
reading is not above that threshold then the system goes to sleep by turning the
CPU to its lowest sleep state and the HX711 to the sleep state. This allows for
a lot of power saving as most of the time the scale is not used.

The sleep mode used is the "Power Down Mode" which is the lowest power
consumption state of the MCU. The sleep modes and their power consummations are
shown in figure \ref{img:sleep}

\begin{figure}[h]
  \centering
  \includegraphics[width=1.0\textwidth]{powermodes}
  \caption{Image obtained from \myLink{http://ww1.microchip.com/downloads/en/DeviceDoc/ATmega4808-4809-Data-Sheet-DS40002173A.pdf}{ATMEGA 4809 Datasheet}}
  \label{img:sleep}
\end{figure}

By putting the MCU to sleep we save $10^3$W of power as the
current consumption goes from mA to uA.

The only way to wake up from this sleep mode is using an external pin interrupt
or an Real Time Clock Interrupt. We would have liked to used an external
interrupt that triggered only when there is a mass above 2Kg on the scale,
however, the digital communication did not allow us to do that. Instead we had
to use the on board RTC to send a periodic interrupt to the MCU every 2 seconds
to wake it up and check the mass.

\subsection{Problems Encountered}

There were not many issues faced during the initial build of the system as
everything was documented well. It just took a fair bit of time to come to an
understanding of how load cells work and choose what hardware to use. The
problems occurred after we had to move away from campus due to COVID-19 and did
not have access to the Arduino we were using as it belonged to the lab. We had
to switch to a new Arduino board that did not support the same sleep modes
available. This process was tedious as a lot of testing had to be done to figure
out how to use the RTC to wake up the MCU. In the end we did figure it
out. Further, working on the project when the team had to communicate online was
also difficult as only one of us could see the errors firsthand.

Another problem was that the scale had to be recalibrated each time the scale
was moved. This was due to the fact that the slightest movement of the sensors
placement affected the readings from the HX711 vastly. Since we did not have a
proper mounted fit, the load cells shifted quite easily. This issue was largely
mitigated by taping the sensors to the surface they rested on, and by not moving
the scale without recalibrating it afterwords.

\section{Current Limitations and Potential Improvements}

The biggest limitation of this project is the calibration issue. Currently the
load cells rest on mounts that were made for sensors of a different shape, and
thus they shift very easily. We expect that by 3D printing a custom mount for
the load cells the need for recalibration would be almost entirely removed,
possibly even to the point of making it unnecessary to tare the scale each time
it turns on.

The system could be improved by implementing our long-term goal of storing
readings in a database, and using the data for tracking weight over time or
finding correlations with other health data.

Another notable limitation of our current design is that readings are only
visible on the console. Modifying the hardware design to add an LCD screen would
allow the system to function without the need for the Arduino to be connected to
an external computer. In a similar vein, it would useful to compute a single
measurement instead of just reporting a stream of data from the sensor. In
particular, an algorithm could be made to monitor the stream of data, detect
when it is relatively stable, and only display the average of the stable
measurements.

The power efficiency of the current system is limited by poor foresight in the
hardware design. It might be possible to design the hardware such that the
resistance of the load cells can be amplified and measured as an analogue
voltage. If this were possible, then the Arduino could be configured so that
instead of waking up every few seconds to check if there is a weight on the
scale, the presence of a weight triggers an interrupt that wakes the Arduino
from sleep.

\newpage
\appendix

\section{Proposal}

This is the original proposal that we submitted:

\begin{quote}
  \input {proposal.txt}
\end{quote}

\section{Source Code}
\label{apx:code}

The code for our project is available on
\myLink{https://github.com/SohamKaranjikar/WeightScaleProject}{GitHub} and on
\myLink{https://gitlab.engr.illinois.edu/cs-431-spring-2020/4crprojects/itj2\_sohammk2}{GitLab}. For
convenience, we have reproduced the code below.

\lstinputlisting[language=C]{../../BathroomScale/BathroomScale.ino}


\section{List of Purchased Hardware and Useful Third-Party Software}
\label{apx:links}

\begin{itemize}
  \item{For the controller, we used ultimately ended up using an
    \myLink{https://www.amazon.com/dp/B07MK598QV}{Uno WiFi r2}. Previous
    versions of our code worked with the
    \myLink{https://www.amazon.com/dp/B008GRTSV6}{Uno r3}}, which is typically
    cheaper.
  \item{\myLink{https://www.amazon.com/dp/B079FTXR7Y}{HX711 chip and load
      cells}}
  \item{We used the standard
    \myLink{https://www.arduino.cc/en/main/software}{Arduino IDE}}
  \item{As a dependency, we used
    \myLink{https://github.com/bogde/HX711}{``HX711''}. It can be installed from
    the Arduino IDE; Menu Bar -> Tools -> Manage Libraries. Search for and
    install ``HX711 Arduino Library''}
\end{itemize}

\end{document}
